\section{Methods}

\red{ What we are doing in this paper}
The Wendling model is capable of replicating key features of observed in hippocampal EEG prior to, during and post seizure~\citep{wendling2002epileptic}. Here the estimation of physiologically relevant model parameters from the Wendling model is considered. An unscented Kalman filter is used to estimate the model parameters of interest. For this study the estimation procedure is tested using artificial EEG simulated using the Wendling model. By doing so it is possible to determine how robust the unscented Kalman filter is when estimating the Wendling model.

\subsection{Model Description and Simulation}

\red{Wendling model description.}

The Wendling model describes the aggregate membrane potentials and firing rates produced by different neural populations. The populations are then excited or inhibited by other populations in the model. The net effect of one population on another is determined by a scaling constant termed the connectivity constant. A graphical representation of the model is shown in Figure~\ref{fig: Biological}. In the model four different neural populations are considered. The pyramidal neural population is the generator of the model output, and its aggregate membrane potential is the simulated artificial EEG. Excitatory interneurons excite the pyramidal neurons. Slow and fast inhibitory interneurons inhibit the pyramidal neural population. Further, the pyramidal neural population excites the excitatory, slow and fast inhibitory populations. Lastly, slow inhibitory interneurons inhibit the fast inhibitory neural population. The effect of each neural population on the other is scaled by a connectivity constant. This connectivity constant accounts for the number of neurons within each population.  Lastly, a stochastic input to the model is added to account for the effect of afferent pyramidal neurons on the modeled neural mass.
\begin{figure}  %%%%%%%%%%%%%%%%%%%%%%%%%%%%%%%%%%%%%%%
	\centering
		\includegraphics{jpg/Biological_Model_Description.jpg}
	\caption{Graphical Descption of the Wendling model.}
	\label{fig: Biological}
\end{figure}

\red{Mathematical functions used in the model}
The Wendling model consists of two primary structures. The first structure is a sigmoid function which converts an aggregate membrane potential to an average firing rate (Equation~\ref{eqn: Sigmoid}). In equation~\ref{eqn: Sigmoid} $g(v(t))$ is the firing rate, $g_{max}$ is the maximum firing rate, $r$ is the sigmoid gradient and $v_{0}$ is the membrane potential at which $0.5g_{max}$ is reached. The second structure is an average firing rate to aggregate membrane potential kernel (Equations~\ref{eqn: Convert}-\ref{eqn: Kernel}). In equations~\ref{eqn: Convert}-\ref{eqn: Kernel} $v_k(t)$ is the aggregate membrane potential, $h_{k}(t)$ is kernel which converts firing rates into membrane potentials, $\theta_{k}(t)$ is the synaptic gain and $\tau_{k}$ is the time constant. $h_{k}(t)$ is a delayed exponential decay function (Figure~\ref{fig: FR2PSP_final}).\begin{align}%%%%%%%%%%%%%%%%%%%%%%%%%%%%%%%%%%%%%%%%%%%%%
\label{eqn: Sigmoid}
g(v(t)) = \frac{g_{max}}{1+exp(r(v(t)-v_{0}))}\\
\label{eqn: Convert}
v_{k}(t) = h_{k}(t)*g(v(t))\\
\label{eqn: Kernel} 
h_{k}(t) = \frac{\theta_{k}(t)t}{\tau_{k}}exp\left(-\frac{t}{\tau_{k}}\right).
\end{align} Here the subscript $k$ is used to indicate that each neural population is described with a different synaptic gain and time constant. The synaptic gains are described by $\theta_{k}$ as they are the parameters that are altered in order to simulate artificial seizure EEG (Figure~\ref{fig: SeizureSim}). Further, the synaptic gains will be the variables that are estimated in this study. In this description of the Wendling model numerous model parameters are considered to be stationary, this includes the maximum firing rate~($g_{max}$) and time constants~($\tau_{k}$). 
\begin{figure}%%%%%%%%%%%%%%%%%%%%%%%%%%%%%%%%%%%%%%%%%%%%%%%%%
	\centering
		\includegraphics{pdf/FR2PSP_final.pdf}
	\caption{Firing rate to aggregate membrane potential converter.}
	\label{fig: FR2PSP_final}
\end{figure} 
\begin{figure}%%%%%%%%%%%%%%%%%%%%%%%%%%%%%%%%%%%%%%%%%%%%%%%%%%%%%%%%%%%%%
	\centering
		\includegraphics{../../Images/Images/pdf/Boon_SF_512_FO_10_HC_3_LC_100Wendling_Output.pdf}
	\caption{Artificial seizure simulated using the wendling model.}
	\label{fig: SeizureSim}
\end{figure}
Equations~\ref{eqn: Convert}-\ref{eqn: Kernel} can be converted into a set of two differential equations \begin{align}%%%%%%%%%%%%%%%%%%%%%%%%%%%%%%%%%%%%%%%%%%%%%%%%%%%%%%%%%%%
\label{eqn: FR2PSP1}
\dot{v}_{k}(t)&= z_{k}(t)\\
\label{eqn: FR2PSP2}
\dot{z}_{k}(t)&=\theta_{k}(t)\tau_{k}n_{k}u_{k}(t)-2\tau_{k}z_{k}(t)-\tau_{k}^{2}v_{k}(t).
\end{align} Here $v_{k}(t)$ is the average membrane potential and $z_{k}(t)$ is its derivative. $\theta_{k}(t)$ and $\tau_{k}$ are the specific neural populations synaptic gain and time constant. Lastly, $u_{k}(t)$ is the firing rate to the specific neural population considered and $n_{k}$ is a constant used to specify connectivity.

\red{Full mathematical description of the model}
Using equations~\ref{eqn: FR2PSP1}-\ref{eqn: FR2PSP2} the model can be described by a set of eight stochastic differential equations \begin{align}%%%%%%%%%%%%%%%%%%%%%%%%%%%%%%%%%%%%%%%
dv_{po}(t)&= z_{po}(t)dt\\
dz_{po}(t)&=(\theta_{p}(t)\tau_{p}n_{p}g(v_{p}(t))-2\tau_{p}z_{po}(t)-\tau_{p}^{2}v_{po}(t))dt\\
dv_{eo}(t)&= z_{eo}(t)dt\\
dz_{eo}(t)&=(\theta_{e}(t)\tau_{e}(\mu +n_{e}g(v_{e}(t))-2\tau_{e}z_{eo}(t)-\tau_{e}^{2}v_{eo}(t))dt + \theta_{e}(t)\tau_{e}\epsilon(t)dW\\
dv_{sio}(t)&= z_{sio}(t)dt\\
dz_{sio}(t)&=(\theta_{si}(t)\tau_{si}n_{si}g(v_{si}(t))-2\tau_{si}z_{sio}(t)-\tau_{si}v_{sio}(t))dt\\
dv_{fio}(t)&= z_{fio}(t)dt\\
dz_{fio}(t)&=(\theta_{fi}(t)\tau_{fi}n_{fi}g(v_{fi}(t))-2\tau_{fi}z_{fio}(t)-\tau_{fi}v_{fio}(t))dt.
\end{align} $dW$ represents a Wiener process and is required as $\epsilon(t)\longmapsto N(0,\sigma)$. Here $\sigma$ and $\mu$ describe the mean and variance of the stochastic model input. Further, $v_{ko}(t) $ and $z_{ko}(t) $ represent the membrane potential produced by each neural population. $v_{k}(t) $ and $z_{k}(t) $ are the inputs to each neural population, and are considered to be the membrane potential of the specific population. $k$ takes the values of $p$, $e$, $fi$ and $si$ representing pyramidal, excitatory, and slow and fast inhibitory populations, respectively. Therefore $v_{p}(t) $ is the output of the model. All $v_{k}(t) $ can be described in terms of $v_{ko}(t) $ as follows \begin{align}%%%%%%%%%%%%%%%%%%%%%%%%%%%%%%%%%%%%%%%%%%%%%%%%%%%%%%%%%%%%%%%%
v_{p}(t) &= v_{eo}(t)-v_{sio}(t)-v_{fio}(t)\\
v_{e}(t) &= c_{1}v_{po}(t)\\
v_{si}(t) &= c_{3}v_{po}(t)\\
v_{fi}(t) &= c_{5}v_{po}(t)-c_{6}v_{sio}(t),
\end{align} where $c_{1}$, $c_{3}$ and $c_{5}$ represent the connectivity strength from pyramidal to excitatory, slow inhibitory and fast inhibitory populations, respectively. $c_{6}$ represents the connectivity strength from the slow to the fast inhibitory population. Lastly, all $n_{k}$ can be defined as connectivity constants \begin{align}%%%%%%%%%%%%%%%%%%%%%%%%%%%%%%%%%%%%%%%%%%%
n_{p} &=1\\
n_{e} &=c_{2}\\
n_{si} &=c_{4}\\
n_{fi} &=c_{7}.
\end{align} Here $c_{2}$, $c_4$ and $c_7$ represent the connectivity strength from excitatory, slow inhibitory and fast inhibitory to the pyramidal population.

\red{Simulation of model}
This set of continuous stochastic differential equations is discretised using Euler-Mariyama's method, to simulate the artificial EEG. \begin{align}
v_{po,t+T}&=v_{po,t}+Tz_{po,t}\\
z_{po,t+T}&=z_{po,t}+T(\theta_{p,t}\tau_{p}n_{p}g(v_{p,t})-2\tau_{p}z_{po,t}-\tau_{p}^{2}v_{po,t})\\
v_{eo,t+T}&=v_{eo,t}+Tz_{eo,t}\\
z_{eo,t+T}&=z_{eo,t}+T(\theta_{e,t}\tau_{e}(\mu +n_{e}g(v_{e,t})-2\tau_{e}z_{eo,t}-\tau_{e}^{2}v_{eo,t})) + \sqrt{T}\theta_{e,t}\tau_{e}\epsilon_{t}\\
v_{sio,t+T}&=v_{sio,t}+Tz_{sio,t}\\
z_{sio,t+T}&=z_{sio,t}+T(\theta_{si,t}\tau_{si}n_{si}g(v_{si,t})-2\tau_{si}z_{sio,t}-\tau_{si}v_{sio,t})\\
v_{fio,t+T}&=v_{fio,t}+Tz_{fio,t}\\
z_{fio,t+T}&=z_{fio,t}+T(\theta_{fi,t}\tau_{fi}n_{fi}g(v_{fi,t})-2\tau_{fi}z_{fio,t}-\tau_{fi}v_{fio,t}).
\end{align} Here $T$ is the period between solutions. The static parameter values are demonstrated in  Tables~\ref{tab: Static}-\ref{tab: Connectivity}. The variance of the input ($\sigma$) is specified such that 99.7\% of realisations drawn from the Gaussian distribution fall within the specified maximum and minimum firing rate. In this case, the firing rate limits are set to 30 and 150. For the cases where the realisations from the Gaussian distribution are not contained within the limits specified, the specific sample of interest is redrawn from the same Gaussain distribution until the firing rate falls within the specified range. In Table~\ref{tab: Static} the parameters $\theta_{p,t}$, $\theta_{e,t}$, $\theta_{si,t}$ and $\theta_{fi,t}$ are not specified as these parameters will vary for different simulations. However,for this simulation it is assumed that \begin{align}
\theta_{p,t} = \theta_{e,t}.
\end{align}


\singlespacing
\small
\begin{center}
	\begin{table}
			\caption{Static Model Parameters}
		\begin{tabular}{||p{4cm}|p{6cm}|p{1.5cm}|p{1.2cm}||}\hline
			 \textsc{Model parameter}  & \textsc{Physical description} & \textsc{Value} & \textsc{Units}  \\\hline\hline
			 $\tau_{p}$ & Time constant for pyramidal neurons & 100 & $s^{-1}$\\\hline
			 $\tau_{e}$ & Time constant for excitatory neurons & 100 & $s^{-1}$\\\hline
			 $\tau_{si}$ & Time constant for slow inhibitory neurons & 35 & $s^{-1}$\\\hline
			 $\tau_{fi}$ & Time constant for fast inhibitory neurons & 500 & $s^{-1}$\\\hline
			 $c$ & Connectivity constant & 135 & NA\\\hline
			 $g_{max}$ & Maximum firing rate & 5 & Hz \\\hline
			 $v_{0}$ & PSP for which 50\% firing rate is achieved & 6 & $mV^{-1}$\\\hline
			 $r$ & Gradient of sigmoid function & 0.56 & NA \\\hline
			 $\mu$ & Input mean firing rate & 90 & $Hz$\\\hline
			 $\sigma$ & Variance of input firing rate & 15 & $Hz$\\\hline\hline 
		\end{tabular}
		\label{tab: Static}
	\end{table}
\end{center}

\begin{center}
	\begin{table}
			\caption[Static Model Parameters: Connectivity]{Static model parameters: Connectivity. $C$ is the connectivity constant specified in Table~\ref{tab: Static}. Terms in brackets indicate the direction in which the constant affects the system. Here P, E, SI and FI represent populations of pyramidal neurons and excitatory, slow inhibitory and fast inhibitory interneurons, respectively.}
		\begin{tabular}{||p{4cm}|p{7cm}|p{2cm}||}\hline
			 \textsc{Model parameter}  & \textsc{Physical description} & \textsc{Value}
			   \\\hline\hline
			 $c_{1}$ & Connectivity constant (P - E) & $C$ \\\hline
			 $c_{2}$ & Connectivity constant (E + I - P) & $0.8C$ \\\hline
			 $c_{3}$ & Connectivity constant (P - SI) & $0.25C$  \\\hline
			 $c_{4}$ & Connectivity constant (SI - P)& $0.25C$ \\\hline
			 $c_{5}$ & Connectivity constant (P - FI) & $0.3C$ \\\hline
			 $c_{6}$ & Connectivity constant (SI - FI) & $0.1C$ \\\hline
			 $c_{7}$ & Connectivity constant (SI - P) & $0.8C$ \\\hline\hline
		\end{tabular}
		\label{tab: Connectivity}
	\end{table}
\end{center}

\subsection{Estimation}

To begin we define a generic nonlinear system where the state matrix and observer function are some function of the staes such that \begin{align}
\dot{x}(t) = A(x(t),\theta(t)) + B(u(t)) + n(t)
y(t)  = C(x(t)) +r(t),
\end{align}
The unscented Kalman filter consists of two steps a prediction (model propogation) and correction step. In the prediction step model states are propagated through the system

The estimation technique considered in this paper is that of the unscented Kalman filter. The unscented Kalman filter is capable of estimating both states and parameters (dual estimation). This is achieved by augmenting parameters to the state matrix. The model parameters that are estimated are then assumed to have trivial dynamics such that \begin{align}
E(\theta_{k,t+T}) = E(\theta_{k,t}),
\end{align} where E(\cdot) is the expectation.


%The estimation method used was originally described by Kalman, and later developed by Ulah for nonlinear parameter estimation.
%
%The method developed by Ulah makes use of two structures, and unscented transform to allow the nonlinearity of the model to be retained and the kalman update process to update the states and parameters based on observations.
%
%To allow for parameter estimation the parameters of interest are considered to be slow states. That being they vary slowly compared to the model states.
%
%These states are therefore augmented to the normal state matrix.
%
%The unscented transform is described by the following set of equations.
%
%The variable describes the standard deviation of the parameter.
%
%The square root function is a matrix square root, which is achieved using the cholesky matrix decomposition.
%
%The resultant covariance and mean and then described using the following set of equations.
%
%Note that similar to the unscented transform there are weights assigned to particular iterations of the unscented process. This allows the process to specify more or less weight on the propogation of the mean through the model.
%
%Once the covariances and means are determined the model states are updated using the standard kalman filter correction step. 
%
%The equations for the kalman update process are.
%
%Here the variable describes the expected observation noise, and variable y the uncertainty in the model.
%
%Here the observation is the output of the simulated signal.
%
%This uncertainty in the model is used to account for errors induced due to model assumptions.
%
%To determine the accuracy of the estimation process the normalised mean square error of the parameter and state estimates are determined.
%
%This estimation procedure is dependant on the original guess for the model parameters and states.
%
%To make the distance between the actual and initial guess of the state estimates the initial guess is set to the middle of the physiological range specified for each parameter. Standard deviation set to encompass the entire physiological range.
%
%A similar process is done for state guess and standard deviation, equations are...
%
%Model uncertianty is set low. And increased when parameters vary within a single simulation. This allows the parameters to track as their standard deviation decreases over periods when the model tracks well and if uncertainty is low these parameters will not be able to track the actual values precise;y.
%
%To determine the accuracy of the estimation procedure, the initial parameter guess is altered based on percentage from the actual.
%
%Observation noise is increased to determine limits on noise allowable for the model.
%
%Model parameters simulated are altered to determine the effect of the parameters on estimation.
%
%Model parameters are varied within single simulation to determine whether the estimaton procedure can track the varying parameters.
%
%
% Following this the simulations from the extended neural mass model will be presented.
%line
%
%
